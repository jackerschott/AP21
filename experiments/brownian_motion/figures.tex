% !TEX program = lualatex
% !TEX options = -synctex=1 -interaction=nonstopmode -file-line-error -shell-escape -output-directory=%OUTDIR% "%DOC%"

\documentclass[12pt,a4paper,german]{scrartcl}
\usepackage[german]{babel}
\usepackage{amsfonts}
\usepackage{amsmath}
\usepackage{amssymb}
\usepackage{caption}
\usepackage{float}
\usepackage[left=2cm,top=2cm,right=2cm,bottom=2cm]{geometry}
\usepackage{graphicx}
\usepackage[hidelinks]{hyperref}
\usepackage{pgf}

\setlength\parindent{0pt}
\numberwithin{equation}{section}
\pagestyle{empty}

\begin{document}
  \begin{figure}[H]
    \centering
    \input{../../figures/brownian_motion/fig1.pgf}
    \caption{Positionen $(x, y)$ eines Latex-Partikels in Wasser in chronologischer Reihenfolge und im Abstand von jeweils einer Sekunde.}
  \end{figure}

  \begin{figure}[H]
    \centering
    \input{../../figures/brownian_motion/fig2.pgf}
    \caption{Relative Häufigkeiten der räumliche Verschiebungen $\Delta s$ eines Partikels in einer beliebigen Raumrichtung sowie die entsprechende Gaußverteilung dieses Histogramms.}
  \end{figure}

  \begin{figure}[H]
    \centering
    \input{../../figures/brownian_motion/fig3.pgf}
    \caption[short]{Abhängigkeit der gesamten kummulativen quadratischen Verschiebung $r^2$ von der Zeit und lineare Regression des Graphen.}
  \end{figure}
\end{document}
